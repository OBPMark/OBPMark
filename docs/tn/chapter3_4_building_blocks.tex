\section{Benchmark \#4: Processing Building Blocks}
The “Processing Building Blocks” benchmarks are intended to cover processing functions that can be found in applications that are prohibitive to define in a full benchmark. The functions included are derived from on-board processing uses cases in applications such as optical image processing, radar processing, SDR (Software Defined Radio) processing as well as AOCS processing.

\subsection{Benchmark \#4.1: FIR Filters}
In the "FIR Filters" benchmark, one dimensional real and complex data shall be filter with the use of FIR (Finite Impulse Response) filters. The target applications include RF and other on-board signal processing of instrument time signals that require filtering.

On overview of the test profiles, is shown in Table \ref{tab:fir_profiles}.

\begin{table}[!h]
\centering
\begin{tabular}{|l|l|l|l|l|l|}
    \hline
    Profile ID  & Data length     & Input data res. & FIR type  & FFT type  & File Set      \\ \hline 
    \hline
    01	        & 65536 samples	  & 16-bit complex  & 16 taps   & Integer   & fir\_set\_01  \\ \hline
    02	        & 65536 samples	  & 16-bit complex  & 64 taps   & Integer   & fir\_set\_01  \\ \hline
    03	        & 65536 samples	  & 16-bit complex  & 256 taps  & Integer   & fir\_set\_01  \\ \hline
\end{tabular}
\caption{FIR benchmark profiles overview.}
\label{tab:fir_profiles}
\end{table}

\subsubsection{FIR Filter Processing Definition}
TBA.

\subsection{Benchmark \#4.2: FFT Processing}
\label{section:fft}
FFT (Fast Fourier Transform) is a computationally efficient algorithm for DFT (Discrete Fourier Transform), introduced by Cooley and Tukey in 1965.

FFT processing is used in a multitude of on-board applications in optical and radar imaging systems, telecommunications and as well as other RF applications.

The radix-2, decimation in time FFT variant of the algorithm shall be used for all benchmarks.

An overview of the test profiles is shown in Table \ref{tab:fft_profiles}. 

% FIXME FIXME input vs output resolution? 
	
\begin{table}[!h]
\centering
\begin{tabular}{|l|l|l|l|l|l|}
    \hline
    Profile ID  & Data length     & Input data res. & FFT size          & FFT type      & File Set      \\ \hline 
    \hline
    01	        & 65536 samples	  & 16-bit complex  & 1024 pts complex   & Integer      & fft\_set\_01  \\ \hline
    02	        & 65536 samples	  & 16-bit complex  & 2048 pts complex   & Integer      & fft\_set\_01  \\ \hline
    03	        & 65536 samples	  & 16-bit complex  & 4096 pts complex   & Integer      & fft\_set\_01  \\ \hline
    04	        & 65536 samples	  & 16-bit complex  & 1024 pts complex   & Float        & fft\_set\_01  \\ \hline
    05	        & 65536 samples	  & 16-bit complex  & 2048 pts complex   & Float        & fft\_set\_01  \\ \hline
    06	        & 65536 samples	  & 16-bit complex  & 4096 pts complex   & Float        & fft\_set\_01  \\ \hline
\end{tabular}
\caption{FFT benchmark profiles overview.}
\label{tab:fft_profiles}
\end{table}
	
\subsubsection{Complex FFT Definition}
For the related benchmarks, the following definition of the complex FFT shall be used:\\
\\
TBA.

\subsubsection{Real FFT Definition}
For the related benchmarks, the following definition of the real FFT shall be used:\\
\\
TBA.

\subsubsection{Inverse FFT Definition}
For the related benchmarks, the following definition of the inverse FFT shall be used:\\
\\
TBA.
	
\subsection{Benchmark \#4.3: Matrix Multiplication}
TBA.

